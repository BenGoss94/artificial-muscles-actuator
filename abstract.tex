
\section{Abstract}

IPMCs (Ionic Polymer-Metal Composites) have potential for use as artificial muscles or stirrers in many applications. This project is focussed on the mixing capability of three Nafion actuators that stir a container of fluid when a voltage is applied to them. The three different IPMCs tested were placed in a container of water and blue dye. A voltage was then applied to the IPMCs that caused them to display artificial muscle behaviours which then mixed the dye and water. The process of mixing is first quantified and defined. Using various processing techniques, the relevant information was extracted such that it could be investigated using computational analytical methods. A vector field of the particle velocities was produced using Particle Image Velocimetry (PIV) which was then used to analyse the fluid dynamics. By looking at image entropy and probability distribution, the mixing capability of the actuators were investigated further.


\vspace{0.5cm}

\textbf{Key Words}

\bigbreak

IPMC - Ionic Polymer-Metal Composite, PIV - Particle Image Velocimetry,  actuator, mixing, image processing, complement, entropy, vortices, curl.

\vspace{0.5cm}

\textbf{\textit{Acknowledgements}}

\bigbreak

\textit{As a group, we wish to acknowledge in particular Dr. Rossiter for guiding us throughout the project.}