
\begin{titlepage}
\begin{center}

% Upper part of the page. The '~' is needed because \\
% only works if a paragraph has started.

\textsc{\Huge University of Bristol}\\[1.5cm]


\textsc{\Large Mathematical and Data Modelling 3}\\[0.5cm]

\textsc{\large Project 3}\\[0.5cm]

% Title
\HRule \\[0.4cm]
{ \huge \bfseries Artificial Muscle Fluid Flow \\[0.4cm] }

\HRule \\[1.5cm]

% Author and supervisor
\begin{minipage}{0.4\textwidth}
\begin{flushleft} \large
\emph{Group D3}\\
Joe Hesketh\\
Ben Goss\\
Marcin Gorny\\
Alice Haynes\\
\end{flushleft}
\end{minipage}
\begin{minipage}{0.4\textwidth}
\begin{flushright} \large
\emph{Supervisor:} \\
Dr. Jonathan Rossiter
\end{flushright}
\end{minipage}
\end{center}

\vspace{1cm}
\center
\begin{minipage}{0.9\textwidth}

\textbf{Abstract}

\textbf{IPMCs (Ionic Polymer-Metal Composites) have potential for use as artificial muscles or stirrers in many applications. This project is focused on the mixing capability of three sizes of Nafion actuator that stir a container of fluid when a voltage is applied to them. The three different IPMCs tested were placed in a container of water and blue dye then filmed as a voltage was applied to the IPMCs via a capacitor. This caused oscillatory bending motion of the actuators that mixed the dye and water. In this report, the process of mixing is first quantified and defined. Various processing techniques are explained by which relevant information can be extracted from the video data such that it can be investigated using computational analytical methods. A vector field of the particle velocities, produced using Particle Image Velocimetry (PIV), is described and used to analyse the fluid dynamics. Image-entropy of the video frames is also discussed as a method for quantifying the mixing capability of the actuators and an attempt is made to measure the rate of mixing for each actuator by modelling the frames as probability distributions and using statistical tests.}


\vspace{1cm}

\textbf{Key Words}


IPMC - Ionic Polymer-Metal Composite, PIV - Particle Image Velocimetry,  actuator, mixing, image processing, complement, entropy, vortices, curl.

\vspace{0.5cm}


\textbf{Acknowledgements}

\textit{As a group, we wish to acknowledge in particular Dr. Rossiter for guiding us throughout the project.}

\end{minipage}



\vfill

\begin{center}
% Bottom of the page
{\large \today}

\end{center}
\end{titlepage}